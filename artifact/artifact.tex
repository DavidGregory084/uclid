\documentclass[runningheads]{llncs}

\usepackage{times} % uses less space
\usepackage{graphicx}
\usepackage{todonotes}
\usepackage{tikz}
\usepackage{listings}
\usepackage{algorithm2e}
\usepackage{enumitem}
\usepackage{wrapfig}
\usepackage{url}
\usepackage{ctable} % for \specialrule command
\usepackage{multirow}
\usepackage{booktabs}
\usepackage{float}
\usepackage{wrapfig} % for wrapping text around tables and figures
\usepackage{caption}
\usepackage{comment}
\usepackage{amsmath}
\usepackage{amssymb}
\usepackage{dirtree}
% \usepackage{subcaption}
\definecolor{darkgreen}{rgb}{0.0, 0.3, 0.0}
\definecolor{codegreen}{rgb}{0,0.4,0}
\definecolor{codegray}{rgb}{0.5,0.5,0.5}
\definecolor{codepurple}{rgb}{0.58,0,0.82}
\definecolor{darkmagenta}{rgb}{0.55,0,0.55}
\definecolor{backcolour}{rgb}{1.0,1.0,1.0}

\newcommand{\uclid}{UCLID5\xspace}
\newcommand{\boogie}{Boogie\xspace}
\newcommand{\zthree}{Z3\xspace}
\newcommand{\smtlib}{SMT-LIB\xspace}
\newcommand{\llamalib}{Synth-Lib\xspace}
\newcommand{\sygusif}{SyGuS-IF\xspace}
\newcommand{\tap}{{\textsc{TAP}}\xspace}
\newcommand{\keyword}[1]{\textcolor{darkmagenta}{\texttt{#1}}}
\newcommand{\ident}[1]{\textcolor{black}{\texttt{#1}}}
\newcommand{\keywordbf}[1]{\textcolor{darkmagenta}{\texttt{{#1}}}}
\newcommand{\proofcmd}[1]{\textcolor{blue}{\texttt{{#1}}}}
\newcommand{\codelike}[1]{\texttt{#1}}
\newcommand{\hbrel}{{\color{blue} \mathsf{hb}}}


\lstdefinelanguage{uclid}{
  sensitive = true,
  keywords={module, forall, exists, Lambda, if, else, assert, assume, havoc,
            for, range, skip, case, esac, init, next, control, function, procedure, oracle,
            returns, call, define, type, var, input, output, const, property,
            invariant, synthesis, grammar, requires, ensures, modifies, instance, axiom, 
            enum, record, integer, boolean, float, true, false, finite_forall, finite_exists, group},
  numbers=left,
  numberstyle=\footnotesize,
  stepnumber=1,
  numbersep=8pt,
  showstringspaces=false,
  breaklines=true,
  frame=top,
  comment=[l]{//},
  morecomment=[s]{/*}{*/},
}

\lstdefinelanguage{smt}{
  sensitive = false,
  keywords={declare, fun, synth, check, assert, sat, blocking, define, constraint},
  numbers=left,
  numberstyle=\footnotesize,
  stepnumber=1,
  numbersep=8pt,
  showstringspaces=false,
  breaklines=true,
  frame=top,
  comment=[l]{;},
}



\lstdefinestyle{uclidstyle}{
  backgroundcolor=\color{backcolour},
  commentstyle=\color{codegreen},
  keywordstyle=\color{magenta},
  numberstyle=\tiny\color{codegray},
  stringstyle=\color{codepurple},
  basicstyle=\footnotesize,
  breakatwhitespace=false,
  basicstyle=\bf\scriptsize\ttfamily,
  breaklines=true,
  captionpos=b,
  keepspaces=true,
  numbers=left,
  numbersep=5pt,
  showspaces=false,
  showstringspaces=false,
  showtabs=false,
  tabsize=2,
  frame=shadowbox
}

\newcommand{\addcite}[1]{\todo[linecolor=blue,backgroundcolor=blue!25,bordercolor=blue]{cite: #1}}

\newcommand{\mat}[1]{{#1}}
\newif\ifcontrol
\controltrue

\begin{document}
%
\title{Artifact for Paper \\
\uclid: Multi-Modal Formal Modeling, Verification, and Synthesis}
%\title{\uclid}
%
\titlerunning{\uclid}
\authorrunning{Anon}
% If the paper title is too long for the running head, you can set
% an abbreviated paper title here
%
\author{Elizabeth Polgreen \inst{1,2}
 \and 
Kevin Cheang\inst{1}
\and
Pranav Gaddamadugu\inst{1}
\and
Adwait Godbole \inst{1}
\and
Kevin Laeufer\inst{1}
\and
Shaokai Lin\inst{1}
\and
Yatin A.~Manerkar\inst{1,3}
\and
Federico Mora\inst{1}
\and ~~~~~~~~~~~~
Sanjit A.~Seshia\inst{1}}
%

\institute{UC Berkeley \and
University of Edinburgh \and
University of Michigan
}

\maketitle

\newcommand{\artcommand}[1]{\texttt{#1}}
\newcommand{\uclidcmd}{\textbf{{UCLID5}}~}

\section{Setup}

The submission tarball has the following structure:
\dirtree{%
.0 .
.1 License.txt.
.1 Readme.txt.
.1 artifact.pdf.
.1 packages.
.2 sbt\_1.5.5\_all.deb.
.1 uclid.
.2 License.txt.
.2 Readme.txt.
.2 tool\_paper\_examples.
.3 uclid.
.3 z3.
.4 solver files.
.3 delphi.
.4 solver files.
.3 cvc4.
.4 solver files.
.3 Control.
.4 associated files.
.3 Keystone.
.4 associated files.
.3 OperAxUhb.
.4 associated files.
.3 TrainSystem.
.4 associated files.
.3 Readme.md.
.3 fib.ucl.
.2 src.
.2 <others>.
}

\paragraph{}
The \artcommand{packages} directory contains the \artcommand{sbt} Debian package which is required to build \uclidcmd from source. This is completely optional and it is not required for running \uclidcmd using the binaries that we have provided.
The \artcommand{uclid} directory is a snapshot of the \uclidcmd GitHub repository at commit \texttt{add commit}. The \artcommand{tool\_paper\_examples} folder is self contained for running the tool on the examples with the provided binaries (see Running with prebuilt binaries below).
To begin, copy the \artcommand{packages} and \artcommand{uclid} directories into the \artcommand{\$HOME} directory of the VM.


\paragraph{Running with prebuilt binaries (recommended)}

The \artcommand{tool\_paper\_examples} directory is self contained with prebuilt \uclidcmd and external solver binaries as well as the relevant examples mentioned in the paper. Please change move into this directory and follow the instructions given in Section \ref{sec:running}.


\paragraph{Building from source} The supplied \artcommand{sbt} Debian package is only required if you wish to build \uclidcmd from source. Start by installing \artcommand{sbt} by running \artcommand{sudo dpkg -i sbt\_1.5.5\_all.deb} from the \artcommand{packages} directory. Then from the \artcommand{uclid} directory, follow the instructions given at \url{https://github.com/uclid-org/uclid#compiling-uclid5}. After doing so, you can either build the \uclidcmd binary or run examples directly from within the \artcommand{}{sbt} prompt.


\section{Running the examples}
\label{sec:running}

Move into the \artcommand{tool\_paper\_examples} directory. This directory consists of the following:

\begin{itemize}
    \item The prebuilt binary for the tool \artcommand{uclid}
    \item The prebuilt binaries for the external solvers \artcommand{z3}, \artcommand{cvc4} and \artcommand{delphi} in their respective directories
    \item Four test-examples directories: \artcommand{Fib}, \artcommand{Control},
    \artcommand{TrainSystem},
    \artcommand{OperAxUhb}, and
    \artcommand{Keystone}
\end{itemize}


To run example \artcommand{examplename}, run the script \artcommand{./run\_<examplename>} from the \artcommand{tool\_paper\_examples} directory.
We now describe each of these examples in more detail:

\newcommand{\excommand}[1]{$\mathsf{#1}$}

\subsection{\excommand{Fib}}
This example corresponds to Figure 4 in our submitted paper. It demonstrates how to use
UCLID5's synthesis syntax.

The model in fib.ucl represents a simple Fibonacci sequence with a partial
proof of the property that the numbers in the sequence always increase. The
proof is partial because the property is not inductive. UCLID5 is able to
complete the proof using a syntax-guided synthesis engine. \textbf{Running
run\_Fib.sh should take around 10 seconds.}

\textbf{This should take around 10 seconds.}

\subsection{\excommand{Control}}
This example corresponds to Figure 5 in our submitted paper. It demonstrates how to use
UCLID5's integration with oracles: 
it uses the oracle \artcommand{is\_stable}, provided in the 
\artcommand{Control} sub-directory.
The model in test-control.ucl represents a Linear Time Invariant system with two state variables. The system is specified using the following matrices:
\begin{align*}
    A &= [0.901224922471, 0.000000013429; 0.000000007451, 0.000000000000] \\
    B &= [128, 0]
\end{align*}
The UCLID model finds a two values for the controller: k0 and k1.
The invariants specify that the controller should stabilize the system (i.e., the eigenvalues should fall within the unit circle, whilst the system states remain within safe bounds up to a finite unrolling bound. 
For more example controllers see \url{https://ssvlab.github.io/dsverifier/dssynth-toolbox/index.html}. 
\subsection{\excommand{OperAxUhb}}

This example corresponds to Figure 8 in our submitted paper, namely the combined operational-axiomatic model of a microarchitecture whose pipelines fetch instructions in program order.
The example code contains \texttt{uhb\_common.ucl}, the embedding of $\mu$spec in \uclid (parts of which are shown in Figure 7 of our submitted paper).
It also contains the code of the model depicted in Figure 8\footnote{Note that the \texttt{fifoFetch} axiom differs very slightly from Figure 8 of our paper due to a small inaccuracy in the submitted paper's code.} of our paper in \texttt{operAx.ucl}.
The \texttt{operAx.ucl} file also contains two properties (\texttt{FetchHBNextExecute} and \texttt{WBFifo}). The control block of \texttt{operAx.ucl} checks these properties using bounded model checking for traces up to size 5.
The first property (\texttt{FetchHBNextExecute}) is maintained by a combination of the operational and axiomatic constraints, and so will always be true.
The second property (\texttt{WBFifo}) is not maintained by the model, and the \texttt{bmc} command duly detects a counterexample trace of length 5 for it.

\textbf{This should take around ?.}

\subsection{\excommand{TrainSystem}}
This example corresponds to Figure 9 and 10 in our submitted paper. It demonstrates the hybrid approach of combining operational and axiomatic modeling using UCLID5.

The model describes a train system written in Lingua Franca (LF), a polyglot coordination language for building deterministic reactive systems. The source code is in \texttt{TrainSystem.lf}. The state transition of the model takes an operational approach by using the \texttt{init} and \texttt{next} keywords in UCLID5, while the semantics of LF are specified using a set of axioms. The model shows a flaw in the train system under verification, by exposing that a bad state, ``the train moves while the door is open,'' is reachable. UCLID5 illustrates the flaw by returning a counterexample to the user.

\textbf{This should take around 10 seconds.}

\subsection{\excommand{OperAxUhb}}
\input{example_OperAxUhb.tex}

\textbf{This should take around 10 seconds.}

\subsection{\excommand{Keystone}}
\input{example_Keystone.tex}

\textbf{This should take around 1 hour.}

\end{document}

